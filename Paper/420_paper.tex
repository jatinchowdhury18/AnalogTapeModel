% !TeX program = pdflatex
% !BIB program = bibtex
% Template LaTeX file for DAFx-19 papers
%
% To generate the correct references using BibTeX, run
%     latex, bibtex, latex, latex
% modified...
% - from DAFx-00 to DAFx-02 by Florian Keiler, 2002-07-08
% - from DAFx-02 to DAFx-03 by Gianpaolo Evangelista
% - from DAFx-05 to DAFx-06 by Vincent Verfaille, 2006-02-05
% - from DAFx-06 to DAFx-07 by Vincent Verfaille, 2007-01-05
%                          and Sylvain Marchand, 2007-01-31
% - from DAFx-07 to DAFx-08 by Henri Penttinen, 2007-12-12
%                          and Jyri Pakarinen 2008-01-28
% - from DAFx-08 to DAFx-09 by Giorgio Prandi, Fabio Antonacci 2008-10-03
% - from DAFx-09 to DAFx-10 by Hannes Pomberger 2010-02-01
% - from DAFx-10 to DAFx-12 by Jez Wells 2011
% - from DAFx-12 to DAFx-14 by Sascha Disch 2013
% - from DAFx-15 to DAFx-16 by Pavel Rajmic 2015
% - from DAFx-16 to DAFx-17 by Brian Hamilton 2016
% - from DAFx-18 to DAFx-19 by Dave Moffat 2019
%
% Template with hyper-references (links) active after conversion to pdf
% (with the distiller) or if compiled with pdflatex.
%
% 20060205: added package 'hypcap' to correct hyperlinks to figures and tables
%                      use of \papertitle and \paperauthorA, etc for same title in PDF and Metadata
%
% 1) Please compile using latex or pdflatex.
% 2) If using pdflatex, you need your figures in a file format other than eps! e.g. png or jpg is working
% 3) Please use "paperftitle" and "pdfauthor" definitions below

%------------------------------------------------------------------------------------------
%  !  !  !  !  !  !  !  !  !  !  !  ! user defined variables  !  !  !  !  !  !  !  !  !  !  !  !  !  !
% Please use these commands to define title and author(s) of the paper:
\def\papertitle{Real-time Physical Modelling for Analog Tape Machines}
\def\paperauthorA{Jatin Chowdhury}

% Authors' affiliations have to be set below

%------------------------------------------------------------------------------------------
\documentclass[twoside,a4paper]{article}
\usepackage{dafx_19}
\usepackage{amsmath,amssymb,amsfonts,amsthm}
\usepackage{euscript}
\usepackage[latin1]{inputenc}
\usepackage[T1]{fontenc}
\usepackage{ifpdf}

\usepackage[english]{babel}
\usepackage{caption}
\usepackage{subfig} % or can use subcaption package
\usepackage{color}

\setcounter{page}{1}
\ninept

\usepackage{times}
% Saves a lot of ouptut space in PDF... after conversion with the distiller
% Delete if you cannot get PS fonts working on your system.

% pdf-tex settings: detect automatically if run by latex or pdflatex
\newif\ifpdf
\ifx\pdfoutput\relax
\else
   \ifcase\pdfoutput
      \pdffalse
   \else
      \pdftrue
\fi

\ifpdf % compiling with pdflatex
  \usepackage[pdftex,
    pdftitle={\papertitle},
    pdfauthor={\paperauthorA},
    colorlinks=false, % links are activated as colror boxes instead of color text
    bookmarksnumbered, % use section numbers with bookmarks
    pdfstartview=XYZ % start with zoom=100% instead of full screen; especially useful if working with a big screen :-)
  ]{hyperref}
  \pdfcompresslevel=9
  \usepackage[pdftex]{graphicx}
  \usepackage[figure,table]{hypcap}
\else % compiling with latex
  \usepackage[dvips]{epsfig,graphicx}
  \usepackage[dvips,
    colorlinks=false, % no color links
    bookmarksnumbered, % use section numbers with bookmarks
    pdfstartview=XYZ % start with zoom=100% instead of full screen
  ]{hyperref}
  % hyperrefs are active in the pdf file after conversion
  \usepackage[figure,table]{hypcap}
\fi
\usepackage{cleveref}
\usepackage{tikz}
\usetikzlibrary{dsp,chains}

\DeclareMathAlphabet{\mathpzc}{OT1}{pzc}{m}{it}
\newcommand{\z}{\mathpzc{z}}

\title{\papertitle}

\affiliation{
\paperauthorA \,}
{\href{http://ccrma.stanford.edu}{Center for Computer Research in Music and Acoustics} \\ Stanford University \\ Palo Alto, CA \\ {\tt \href{mailto:jatin@ccrma.stanford.edu}{jatin@ccrma.stanford.edu}}}

\begin{document}
% more pdf-tex settings:
\ifpdf % used graphic file format for pdflatex
  \DeclareGraphicsExtensions{.png,.jpg,.pdf}
\else  % used graphic file format for latex
  \DeclareGraphicsExtensions{.eps}
\fi

\maketitle

\begin{abstract}
For decades, analog magnetic tape recording was the most popular
method for recording music, but has been replaced over the past 30 years first by
DAT tape, then by DAWs and audio interfaces \cite{Kadis}. Despite being replaced
by higher quality technology, many have sought to recreate a "tape" sound
through digital effects, despite the distortion, tape ``hiss'', and other
oddities analog tape produced. The following paper describes the general process
of creating a physical model of an analog tape machine starting from basic
physical principles, then discusses in-depth a real-time implementation
of a physical model of a Sony TC-260 tape machine.
\newline\newline
"Whatever you now find weird, ugly, uncomfortable, and nasty
about a new medium will surely become its signature. CD distortion, the jitteriness
of digital video, the crap sound of 8-bit - all of these will be cherished
and emulated as soon as they can be avoided." -Brian Eno \cite{Eno}.
\end{abstract}

\section{Introduction}
While analog magnetic tape recording (see \cref{TapeMachine}) is rarely used in modern recording
studios, the sound of analog tape is still often sought after by mixing
and mastering engineers. To that end, several prominent audio
plugin manufacturers including Waves\footnote{\url{https://www.waves.com/plugins/j37-tape}},
Universal Audio\footnote{\url{https://www.uaudio.com/uad-plugins/plug-in-bundles/magnetic-tape-bundle.html}},
and U-He\footnote{\url{https://u-he.com/products/satin/}} have created
tape emulating plugins. Unfortunately,
the existing literature on analog tape emulation is somewhat lacking.
While Arnadottir et at. \cite{tapeDelay} and Valimaki et al. \cite{DAFX_book}
describe the emulation of tape echo/delay devices, and Valimaki et al
\cite{disks} describe the emulation of disk-based audio recording media,
we were unable to locate any existing research directly discussing digital
emulation of the magnetisation process, a gap in research that this 
publication intends to fill. That said, Kadis \cite{Kadis} and Camras
\cite{Camras:1987:MRH:27189} discuss musical use of analog tape recorders
in a useful technical manner, and Bertram \cite{1994tmr..book.....B}
gives a in-depth technical description of the physical underpinnings of
analog magnetic recording; this work intends to build on their foundations.
While tape machines also contain electronic circuits that contribute to the 
machine's characteristic sound, this publication only considers processes
that relate directly to tape magnetisation. For readers
wishing to emulate tape machine circuits, a good overview of circuit
modelling techniques can be found in \cite{Yeh}.

\begin{figure}[ht]
    \center
    \includegraphics[width=2.5in]{../Refs/Pictures/sony_tc-260.jpg}
    \caption{\label{TapeMachine}{\it An analog tape recorder (Sony TC 260)}}
\end{figure}
%

\section{Continuous Time System}
Audio recorded to and played back from a tape machine can be thought of as going
through three distinct processors: the record head, tape magnetisation, and the play
head.

\subsection{The Record Head}
For an instantaneous input current $I(t)$, the magnetic field output of the record 
head is given as a function of distance along the tape ($x$), and depth into 
the tape ($y$). Using the Karlqvist medium field approximation, we find
\cite{1994tmr..book.....B}:

\begin{multline}
    H_x(x,y) = \frac{1}{\pi} H_0 \Big(\tan^{-1} \Big(\frac{(g/2) + x}{y} \Big) \\
    + \tan^{-1} \Big(\frac{(g/2) - x}{y} \Big) \Big)
    \label{eq:H_x}
\end{multline}
\begin{equation}
    H_y(x,y) = \frac{1}{2 \pi} H_0 \ln \Big(\frac{((g/2) - x)^2 + y^2}{((g/2) + x)^2 + y^2} \Big)
    \label{eq:H_y}
\end{equation}
%
where $H_x$ and $H_y$ are components of the magnetic field $\vec{H}$,
$g$ is the head gap, and $H_0$ is the deep gap field, given by:

\begin{equation}
    H_0 = \frac{NEI}{g}
\end{equation}
%
where $N$ is the number of turns of wire around the head, and $E$ is the head 
efficiency which can be calculated by:

\begin{equation}
    E = \frac{1}{1 + \frac{l  A_g}{\mu_r g} \int_{core} \frac {d \vec{l}}{A(l)}}
\end{equation}
%
where $A_g$ is the gap area, $\mu_r$ is the core permeability relative to 
free space ($\mu_0$), and $A(l)$ is the cross-sectional 
area at a point $l$ along its circumferential length.

\subsection{Tape Magnetisation}
The magnetisation recorded to tape from a magnetic field can be described
using a hysteresis loop, as follows \cite{1994tmr..book.....B}:

\begin{equation}
    \vec{M}(x,y) = F_{Loop}(\vec{H}(x,y))
\end{equation}
%
where $F_{Loop}$ is a generalized hysteresis function.
\newline\newline
Using the Jiles-Atherton magnetisation model, the following
differential equation describes magnetisation in some direction ($M$)
as a function of the magnetic field in that direction ($H$) \cite{Hysteresis}:

\begin{equation}
    \frac{dM}{dH} = \frac{(1-c) \delta_M (M_{an} - M)}{(1-c) \delta_S k - \alpha (M_{an} - M)} + c \frac{dM_{an}}{dH}
    \label{eq5}
\end{equation}
%
where $c$ is the ratio of normal and anhysteric initial susceptibilities,
$k$ is a measure of the width of the hysteresis
loop, $\alpha$ is a mean field parameter, representing inter-domain
coupling, and $\delta_S$ and $\delta_M$ are given by:

\begin{equation}
    \delta_S = \begin{cases}
        1 & \text{if H is increasing} \\
        -1 & \text{if H is decreasing}
    \end{cases}
\end{equation}
\begin{equation}
    \delta_M = \begin{cases}
        1 & \text{if $\delta_S$ and $M_{an} - M$ have the same sign} \\
        0 & \text{otherwise}
    \end{cases}
\end{equation}
%
$M_{an}$ is the anisotropic magnetisation given by:

\begin{equation}
    M_{an} = M_s L \Big( \frac{H + \alpha M}{a} \Big)
\end{equation}
%
where $M_s$ is the magnetisation saturation, $a$ characterizes the shape
of the anhysteric magnetisation and $L$ is the Langevin function:

\begin{equation}
    L(x) = \coth (x) - \frac{1}{x}
\end{equation}

\subsection{Play Head}
\subsubsection{Ideal Playback Voltage}
The ideal playback voltage as a function of tape magnetisation at a point
$x$ along the tape is given by
\cite{1994tmr..book.....B}:

\begin{equation}
    V(x) =  NWEv \mu_0 \int_{-\infty}^{\infty} dx' \int_{-\delta/2}^{\delta/2} dy' \vec{h}(x' + x, y') \cdot \frac{\vec{M}(x', y')}{dx}
    \label{eq:Vx}
\end{equation}
%
where $N$ is the number of turns of wire, $W$ is the width of the playhead, $E$ is the playhead
efficiency, $v$ is the tape speed, $\delta$ is the thickness of the tape,
and $\mu_0$ is the permeability of free space.
Note that $V(x) = V(vt)$ for constant $v$. $\vec{h}(x, y)$ is defined as:

\begin{equation}
    \vec{h} (x, y) \equiv \frac{\vec{H} (x, y)}{NIE}
    \label{eq12}
\end{equation}
%
where $\vec{H} (x, y)$ can be calculated by \cref{eq:H_x,eq:H_y}.

\subsubsection{Loss Effects}
There are several frequency-dependent loss effects associated with playback,
described as follows \cite{Kadis}:

\begin{equation}
    V(t) = V_0(t) [e^{-kd}] \Big[\frac{1 - e^{-k \delta}}{k \delta} \Big] \Big[\frac{\sin (kg /2)}{kg/2} \Big]
    \label{eq:lossEffects}
\end{equation}
%
for sinusoidal input $V_0(t)$, where $k$ is the wave number, $d$ is the distance between the tape and the playhead,
$g$ is the gap width of the play head, and again $\delta$ is the thickness of the tape. The wave number
is given by:

\begin{equation}
    k = \frac {2 \pi f}{v}
    \label{eq:wavenumber}
\end{equation}
%
where $f$ is the frequency and $v$ is the tape speed.

\section{Digitizing the System}
\subsection{Record Head}
For simplicity, let us assume,
\begin{equation}
    \vec{H}(x,y,t) = \vec{H}(0,0,t)
    \label{eq:spatialApprox}
\end{equation}
%
In this case $H_y \equiv 0$, and $H_x \equiv H_0$. Thus,
\begin{equation}
    H(t) = \frac{NEI(t)}{g}
    \label{eq15}
\end{equation}
%
or,
\begin{equation}
    \hat{H}(n) = \frac{NE\hat{I}(n)}{g}
    \label{eq:Hin}
\end{equation}

\subsection{Hysteresis}
Beginning from \cref{eq5}, we can find the derivative of $M$ w.r.t. time,
as in \cite{Hysteresis}:
\begin{equation}
    \frac{dM}{dt} = \frac{\frac{(1-c) \delta_M (M_sL(Q) - M)}{(1-c) \delta_S k - \alpha (M_sL(Q) - M)} \dot{H} + c \frac{M_s}{a} \dot{H} L'(Q)}{1 - c \alpha \frac{M_s}{a} L'(Q)}
    \label{eq:dmdt}
\end{equation}
%
where $Q = \frac{H + \alpha M}{a}$, and

\begin{equation}
    L'(x) = \frac{1}{x^2} - \coth^2(x) + 1
\end{equation}
%
Note that \cref{eq:dmdt} can also be written in the general form for non-linear
Ordinary Differential Equations:
\begin{equation}
    \frac{dM}{dt} = f(t,M,\vec{u})
\end{equation}
where $\vec{u} = \begin{bmatrix}
    H \\
    \dot{H}
    \end{bmatrix}$.
\newline\newline
Using the trapezoidal rule for derivative approximation, we find:

\begin{equation}
    \dot{\hat{H}}(n) = 2\frac{\hat{H}(n) - \hat{H}(n-1)}{T} - \dot{\hat{H}}(n-1)
    \label{eq:hDeriv}
\end{equation}
%
We can use the Runge-Kutta 4th-order method \cite{Yeh} to find an explicit solution
for $\hat{M}(n)$:
\begin{align}
\begin{split}
    k_1 &= T f \Big(n-1, \hat{M}(n-1), \hat{\vec{u}}(n-1) \Big)\\
    k_2 &= T f \Big(n - \frac{1}{2}, \hat{M}(n-1) + \frac{k_1}{2}, \hat{\vec{u}}  \Big(n-\frac{1}{2} \Big) \Big)\\
    k_3 &= T f \Big(n- \frac{1}{2}, \hat{M}(n-1) + \frac{k_2}{2}, \hat{\vec{u}} \Big(n-\frac{1}{2} \Big) \Big)\\
    k_4 &= T f \Big(n, \hat{M}(n-1) + k_3, \hat{\vec{u}}(n) \Big)\\
    \hat{M}(n) &= \hat{M}(n-1) + \frac{k_1}{6} + \frac{k_2}{3} + \frac{k_3}{3} + \frac{k_4}{6}
    \label{eq:Mn}
\end{split}
\end{align}
%
We use linear interpolation to find the half-sample values used to calculate $k_2$ and $k_3$.
Note that many audio DSP systems prefer
lower-order implicit methods such as the trapezoidal rule to solve
differential equations rather than a higher-order method like the
Runge-Kutta method \cite{Yeh}. However in this case, it was found that
the lower-order methods quickly became unstable for high-frequency input,
particularly when the input is modulated by a bias signal (see \Cref{bias}).

\subsubsection{Numerical Considerations} \label{numerical}
To account for rounding errors in the Langevin function for values close to 
zero, we use the following approximation about zero, as in \cite{Hysteresis}:
\begin{equation}
    L(x) = \begin{cases}
        \coth(x) - \frac{1}{x} & \text{for $|x| > 10^{-4}$} \\
        \frac{x}{3} & \text{otherwise}
    \end{cases}
\end{equation}
\begin{equation}
    L'(x) = \begin{cases}
        \frac{1}{x^2} - \coth^{2}(x) + 1 & \text{for $|x| > 10^{-4}$} \\
        \frac{1}{3} & \text{otherwise}
    \end{cases}
\end{equation}
%
Additionally, $\tanh(x)$, and by extension $\coth(x)$ is a
rather computationally expensive operation. With this in mind,
for real-time implementation, we approximate $\coth(x)$ as the
reciprocal of a Gaussian continued fraction for $\tanh(x)$
\cite{MATH}, namely
\begin{equation}
    \tanh(x) = \frac{x}{1 + \frac{x^2}{3 + \frac{x^2}{5 + \frac{x^2}{7}}}}
    \label{eq:continuedFrac}
\end{equation}

\begin{figure}[ht]
    \center
    \includegraphics[width=3in]{../Simulations/Hysteresis/Sim2-M_H.png}
    \caption{\label{HysteresisSim}{\it Digitized Hysteresis Loop Simulation}}
\end{figure}
%
\subsubsection{Simulation}
The digitized hysteresis loop was implemented and tested offline
in \texttt{Python}, using the constants $M_s$, $a$, $\alpha$, $k$,
and $c$ from \cite{JilesAtherton1986}. For a sinusoidal input signal
with frequency 2kHz, and varying amplitude from 800 - 2000 Amperes per
meter, \cref{HysteresisSim} shows the Magnetisation output.

\subsection{Play Head}
By combining \cref{eq:Vx} with \cref{eq12,eq15}, we get:
\begin{equation}
    V(t) =  NWEv \mu_0  g M(t)
\end{equation}
%
or,
\begin{equation}
    \hat{V}(n) =  NWEv \mu_0  g \hat{M}(n)
    \label{eq:Vout}
\end{equation}
%
\subsubsection{Loss Effects} \label{Loss_Effects}
In the real-time system, we model the playhead
loss effects with an FIR filter, derived by
taking the inverse DFT of the
loss effects described in \cref{eq:lossEffects}.
It is worth noting that as in \cref{eq:wavenumber},
the loss effects, and therefore the FIR filter
are dependent on the tape speed.
\newline\newline
The loss effects filter was implemented and
tested offline in \texttt{Python} with tape-head 
spacing of 20 microns, head gap width of 5 microns, 
tape thickness of 35 microns, and tape speed of 15 ips.
The following plot shows the results of the simulation,
with a filter order of 100.
\begin{figure}[ht]
    \center
    \includegraphics[width=3in]{../Simulations/LossEffects/Loss_Effects.png}
    \caption{\label{lossEffectsSim}{\it Frequency Response of Playhead Loss Effects}}
\end{figure}
%

\section{Tape and Tape Machine Parameters}
In the following sections, we describe the implementation of
a real-time model of a Sony TC-260 tape machine, while attempting
to preserve generality so that the process can be repeated for any
similar reel-to-reel tape machine.

\subsection {Tape Parameters}
A typical reel-to-reel tape machine such as the Sony TC-260 uses 
Ferric Oxide ($\gamma F_2O_3$) magnetic tape. The following
properties of the tape are necessary for the tape hysteresis
process \cref{eq:dmdt}:
\begin{itemize}
\item Magnetic Saturation ($M_s$): For Ferric Oxide tape
the magnetic saturation is $3.5e5$ (A/m) \cite{jilesBook}.
\item Hysteresis Loop Width ($k$): For soft materials, $k$ can be approximated
as the coercivity, $H_c$ \cite{Jiles1992}. For Ferric Oxide, $H_c$ is approximately
$27$ kA/m \cite{jilesBook}.
\item Anhysteric magnetisataion ($a$): Knowing the coercivity and remnance magnetism of Ferric Oxide
\cite{jilesBook}, we can calculate $a$ = 22  kA/m by the method described in
\cite{Jiles1992}.
\item Ratio of normal and hysteris initial susceptibilities ($c$): From \cite{Jiles1992}, $c$ = 1.7e-1.
\item Mean field parameter ($\alpha$): From \cite{Jiles1992}, $\alpha$ = 1.6e-3.
\end{itemize}

\subsection{Tape Machine Parameters}
\subsubsection {Record Head}
To determine the magnetic field output of the
record head using \cref{eq:Hin}, the following parameters 
are necessary:
\begin{itemize}
\item Input Current ($\hat{I} (n)$): For the Sony TC-260
the input current to the record head is approximately
0.1 mA peak-to-peak \cite{RefManual}.
\item Gap Width ($g$): The gap width for recording heads
can range from 2.5 to 12 microns \cite{Kadis}.
\item Turns of wire ($N$): The number of turns of wire
is typically on the order of 100 \cite{1994tmr..book.....B}.
\item Head Efficiency ($E$): The head efficiency is typically
on the order of 0.1 \cite{1994tmr..book.....B}.
\end{itemize}
%
These values result in a peak-to-peak magnetic field
of approximately 5e5 A/m.

\subsubsection{Play Head}
Similar to the record head, the following parameters
are needed to calculate the output voltage using
\cref{eq:Vout,eq:lossEffects} (note that values are only included
here if notably different from the record head):
\begin{itemize}
\item Gap Width ($g$): The play head gap width ranges from
1.5 to 6 microns\cite{Kadis}.
\item Head Width ($W$): For the Sony TC-260, the play head
width is 0.125 inches (note that this is the same as the
width of one track on the quarter-inch tape used by the 
machine) \cite{RefManual}.
\item Tape Speed ($v$): The Sony TC-260 can run at 3.75 inches
per second (ips), or 7.5 ips \cite{RefManual}. Note that many
 tape machines can run at 15 or 32 ips \cite{Kadis}.
\item Tape Thickness ($\delta$): Typical tape that would be used
with the TC-260 is on the order of 35 microns thick \cite{RefManual}.
\item Spacing ($d$): The spacing between the tape and the play
head is highly variable between tape machines. For a typical
tape machine spacing can be as high as 20 microns \cite{Kadis}.
\end{itemize}

\begin{figure}[ht]
    \center
    \includegraphics[width=3in]{../Simulations/Bias/BiasEx.png}
    \caption{\label{Bias}{\it Example of a biased signal}}
\end{figure}
%
\subsection{Tape Bias}
\label{bias}
A typical analog recorder adds a high-frequency "bias"
current to the signal to avoid the "deadzone" effect when the input signal
crosses zero, as well as to linearize the output. The input
current to the record head can be given by
\cite{Camras:1987:MRH:27189}:
\begin{equation}
    \hat{I}_{head}(n) = \hat{I}_{in}(n) + B \cos(2 \pi f_{bias} n T)
    \label{eq:bias}
\end{equation}
%
Where the amplitude of the bias current $B$ is usually
about one order of magnitude larger than the input,
and the bias frequency $f_{bias}$ is well above the
audible range. \Cref{Bias} shows a unit-amplitude,
2 kHz sine wave biased by a 50 kHz sine wave with amplitude 5.
To recover the correct output signal, tape machines use a
lowpass filter, with a cutoff frequency well below the bias
frequency, though still above the audible range \cite{Kadis}.
\newline\newline
For the Sony TC-260, the bias frequency is 55 kHz, with a gain
of 5 relative to the input signal. The lowpass filter used to recover
the audible signal has a cutoff at 24 kHz, though note that due to
the frequency response of the playhead loss effects, the effects
of this filter may be essentially neglible to the real time system.
\cite{RefManual}

\subsection{Wow and Flutter} \label{flutter}
Each tape machine has characteristic timing imperfections
known as ``wow'' and/or ``flutter.'' These imperfections
are caused by minor changes in speed from the motors
driving the tape reels, and can cause fluctuations in
the pitch of the output signal. To characterize these
timing imperfections, we use a method similar to \cite{tapeDelay}:
We recorded a pulse train of 1000 pulses through a TC-260,
then recorded the pulses back from the tape. \Cref{timingSim}
shows a section of a superimposed plot of the original
pulse train against the pulse train recorded from the tape
machine. From this data, we were able to generate a periodic
function that accurately models the timing imperfections of
the TC-260. The process was performed at both 7.5 ips and 3.75
ips. In the real-time system, the timing imperfection model
is used to inform a modulating delay line, to achieve the
signature "wow" effect of an analog tape machine.

\begin{figure}[ht]
    \center
    \includegraphics[width=3in]{../Simulations/TimingEffects/timing_diff_7-5.png}
    \caption{\label{timingSim}{\it Input pulse train superimposed with pulse train recorded from TC-260}}
\end{figure}
%
\begin{figure}[ht]
    \begin{center}
        \begin{tabular}{|| c | c ||}
            \hline
            Record Head Constants & \\
            \hline
            Turns of wire (N) & 100 \\
            Head Efficiency (E) & 0.1 \\
            Record Head Gap (g) & 6 (microns) \\
            \hline
            \hline
            \hline
            Tape Constants & \\
            \hline
            Magnetic Saturation ($M_s$) & 3.5e5 (A/m) \\
            Hysteresis Loop Width (k) & 27 (kA/m) \\
            Anhysteric Magnetisation (a) & 22 (kA/m) \\
            Ratio of magnetic susceptibilities (c) & 1.7e-1 \\
            \hline
            \hline
            \hline
            Play Head Constants & \\
            \hline
            Play Head Width (W) & 0.125 (in) \\
            \hline
        \end{tabular}
    \end{center}
    \caption{\label{constants}{\it Constant values for model implementation}}
\end{figure}
%
\begin{figure*}[ht]
    \center
    \begin{tikzpicture}
        \matrix (m1) [row sep=10mm, column sep=8mm]
        {
                                                                                &
                                                                                &
            \node[dspnodeopen,dsp/label=above]   (b0) {Bias Signal};            &
                                                                                &
                                                                                &
                                                                                &
            \node[dspnodeopen,dsp/label=above]   (b1) {Flutter ($\tau$)};       &
            \node[dspnodeopen,dsp/label=above]   (b2) {Tape Speed ($v$)};       \\
            %---------------------------------------------------------------
            \node[dspnodeopen,dsp/label=left]    (x0) {$x[n]$};              &
            \node[dspsquare]                     (x1) {\upsamplertext{M}};   &
            \node[dspadder]                      (x2) {};                    &
            \node[dspfilter]                     (x3) {$H_{record}$};        &
            \node[dspfilter]                     (x4) {Hysteresis};          &
            \node[dspsquare]                     (x5) {\downsamplertext{M}}; &
            \node[dspsquare]                     (x6) {$z^{-\tau}$};         &
            \node[dspfilter]                     (x7) {$H_{play}(v)$};       &
            \node[dspnodeopen,dsp/label=right]   (x8) {$y[n]$};              \\
        };
        \draw[dspconn] (b0) -- (x2);
        \draw[dspconn] (b1) -- (x6);
        \draw[dspconn] (b2) -- (x7);
        \foreach \i [evaluate = \i as \j using int(\i+1)] in {0,1,...,7}
            \draw[dspconn] (x\i) -- (x\j);
    \end{tikzpicture}
    \caption{\label{flowchart}{\it Flowchart of realtime system: 
                                   M is the oversampling factor,
                                   $H_{record}$
                                   is the transfer function of the record
                                   head, and
                                   $H_{play}$ is the play head transfer
                                   function including loss effects, and a 
                                   de-biasing filter.}}
\end{figure*}
%
\section{Real-Time Implementation}
We implemented our physical model of the Sony TC-260 as a
VST audio plugin using the JUCE framework. \Cref{flowchart}
shows the signal flow of the system in detail. We allow the user
to control parameters in real-time including the tape speed, bias gain, gap
width, tape thickness, tape spacing, and flutter depth.
\newline\newline
Bias modulation is implemeted using \cref{eq:bias}, where bias gain
$B$ and bias frequency $f_{bias}$ are controlled by the user.
\newline\newline
The record head transfer function calculates the record head magnetic
field $H$ from the input current $I$, and is implemented using \cref{eq:Hin},
with constant values shown in \cref{constants}.
\newline\newline
The hysteresis process calculates the tape magnetisation $M$ from the
record head magnetic field $H$, and is implementated using the Runge-Kutta
method described in \cref{eq:Mn}, with constant value defined in \cref{constants}.
\newline\newline
The flutter process is implemented using a modulated delay line as described in
\cref{flutter}, with user controlled modulation depth.
\newline\newline
The play head transfer calculates the play head voltage $V$ from the
tape magnetisation $M$, using \cref{eq:Vout} and the loss effects filter
described in \cref{Loss_Effects}. The gap width $g$, tape speed $v$, tape
thickness $\delta$, and spacing $d$ are controlled by the user, and
other constant values are shown in \cref{constants}.
\newline\newline
C/C++ code for the full real-time implementation is open-source
and is available on GitHub.\footnote{\url{https://github.com/jatinchowdhury18/AnalogTapeModel}}

%
\subsection{Oversampling}
If no oversampling is used, the system will be unstable
for input signal at the Nyquist frequency, due to limitations
of the trapezoid rule derivate approximation used in \cref{eq:hDeriv}.
To avoid this instability, early versions of the real-time
implementation used a lowpass filter with cutoff frequency
just below Nyquist. However, due to aliasing caused by the
nonlinearity of the tape hysteresis model, oversampling is
necessary to mitigate aliasing artifacts \cite{Yeh}. Further,
the system must be able to faithfully recreate not only the
frequencies in the audible range but the bias frequency as
well. Since the TC-260 uses a bias frequency of 55 kHz \cite{RefManual}
and the minimum standard audio sampling rate is 44.1 kHz,
a minimum oversampling factor of 3x is required. However,
since the biased signal is then fed into the hystersis model,
even more oversampling is required to avoid aliasing. With these
considerations in mind, our system uses an oversampling
factor of 16x.

\begin{figure}[ht]
    \center
    \includegraphics[width=2.8in]{../Testing/DeadzoneTest.png}
    \includegraphics[width=2.8in]{../Testing/FlutterTest.png}
    \caption{\label{tests}{\it Testing results for real-time system:
                            sine wave output with no biasing (above),
                            input vs. output pulse train comparison (below).}}
\end{figure}
%
\begin{figure}[ht]
    \center
    \includegraphics[width=2.8in]{../Testing/HysteresisTest.png}
    \includegraphics[width=2.8in]{../Testing/HysteresisTest_Tape.png}
    \caption{\label{tests2}{\it Test results comparing real-time system
                                to Sony TC-260 physical unit:
                                hysteresis loop for real-time system (above),
                                hysteresis loop for TC-260 (below).}}
\end{figure}
%

\subsection{Results}
In subjective testing, our physical model sounds quite convincing,
with warm, tape-like distortion, and realistic sounding
flutter. The high-frequency loss and low-frequency
``head bump'' change correctly at different tape speeds,
and are approximately within the frequency response
specifications of the TC-260 service manual \cite{RefManual}.
When the input to the plugin is silent, the hysteresis processing
of the bias signal produces a very accurate ``tape hiss'' sound.
The distortion and frequency response characteristics
of our model are subjectively very close when compared to
the output of an actual TC-260, though not nearly close enough to 
``fool'' the listener. Additionally, as the bias
gain is lowered, the ``deadzone'' effect appears exactly
as expected \cite{Camras:1987:MRH:27189}.
The largest difference between the model and the physical
machine is the subtle electrical and mechanical noises
and dropouts present in the physical machine, presumably
caused by the age and wear-and-tear of the machine, which
we did not attempt to characterize in our model.
\Cref{tests} shows the results
of tests performed on the real-time system, including
an example of the ``deadzone'' effect, and the timing
irregularities or ``flutter''. \Cref{tests2} shows
a comparison of hysteresis characteristics between
the real-time software model and a physical Sony TC-260
tape machine. Note that some
differences between the two hysteresis loops may be due
to the circuitry of the tape machine that we did not
attempt to model in the real-time system.
Audio examples from the real-time system can be found
online.\footnote{\url{https://ccrma.stanford.edu/~jatin/420/tape/}}

\subsection{Evaluation}
While there is an audible difference between the
real-time software model and a physical Sony
TC-260, the most fundamental aspects of the tape
machine sound including tape saturation, tape hiss, flutter/wow,
and frequency response, are clearly audible and sound
very accurate. The main distinctions between the two
systems can be attributed to the tape machine circuitry
(in particular the TC-260 contains two shelving filters),
as well as mechanical wear of the system, both elements
that were not considered in our model.
\newline\newline
In our opinion, the
strongest proof of the efficacy of our model is that the
model responds accurately to the adjustement of model parameters.
In particular, the hysteresis process
reacted correctly to changes in input gain (saturating for
overdriven input, or fading into tape hiss for underdriven
input), as well as bias gain (saturation for overbiasing, or
``deadzone'' effect for underbiasing).
Additionally, adjusting the loss effect parameters
correctly demonstrated known tape machine phenomena including
head ``bump'' (a resonance at the wavelength of the play
head gap width), and spacing loss (filtering due to the
spacing between the the play head and tape). The reader is
invited to download the plugin (available with the source code)
and evaluate the model for themselves. In conclusion, we believe
that our model successfully approximates the physical tape
recording process, however for those wishing to model a full
tape machine, we suggest using this model in combination with
a model of the tape machine's circuits.

\section{Future Improvements}
\subsection{Spatial Magnetic Effects}
The most obvious improvement to be made for the physical model
is the inclusion of spatial effects of the tape. In particular,
the approximations made in \cref{eq:spatialApprox}, negate any
effects caused by magnetisation along the longitudinal length
of the tape, and into the depth of the tape. Including spatial
effects would involve deriving digital analogues for
\cref{eq:H_x,eq:H_y,eq:Vx}, and re-deriving \cref{eq:Mn}
to take an 2-dimensional magnetic field input at every timestep,
rather than the zero-dimensional input it currently takes. This change
would greatly increase the computational complexity of the system.
At an oversampling rate of 16x,
using just 100 spatial samples would be 1600x more
computationally complex than the current system.

\section{Acknowledgements}
Many thanks to Julius Smith for guidance and support, and
thanks to Irene Abosch for kindly donating her Sony TC-260.

%\newpage
\nocite{*}
\bibliographystyle{IEEEbib}
\bibliography{references}

\end{document}
